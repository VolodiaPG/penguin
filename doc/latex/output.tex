\PassOptionsToPackage{unicode=true}{hyperref} % options for packages loaded elsewhere
\PassOptionsToPackage{hyphens}{url}

\documentclass[a4paper,11pt]{article}
% \usepackage[]{natbib}
\usepackage[numbers]{natbib}
\bibliographystyle{plain-fr}
\usepackage{exptech}

\usepackage{lmodern}
\usepackage{amssymb,amsmath}
  \usepackage{textcomp} % provides euro and other symbols

\usepackage{fancyhdr}


\usepackage{xcolor}
\IfFileExists{xurl.sty}{\usepackage{xurl}}{} % add URL line breaks if available
\IfFileExists{bookmark.sty}{\usepackage{bookmark}}{\usepackage{hyperref}}
\hypersetup{
  pdftitle={Exercices cours de Secu sur les clouds},
  pdfauthor={Clément ; Romain ; Romain ; Maxime ; Volodia },
  pdfborder={0 0 0},
  breaklinks=true}
\urlstyle{same}  % don't use monospace font for urls




\usepackage{graphicx,grffile}
\makeatletter
\def\maxwidth{\ifdim\Gin@nat@width>\linewidth\linewidth\else\Gin@nat@width\fi}
\def\maxheight{\ifdim\Gin@nat@height>\textheight\textheight\else\Gin@nat@height\fi}
\makeatother
% Scale images if necessary, so that they will not overflow the page
% margins by default, and it is still possible to overwrite the defaults
% using explicit options in \includegraphics[width, height, ...]{}
\setkeys{Gin}{width=\maxwidth,height=\maxheight,keepaspectratio}

% Make links footnotes instead of hotlinks:
\DeclareRobustCommand{\href}[2]{#2\footnote{\url{#1}}}


\setlength{\emergencystretch}{3em}  % prevent overfull lines
\providecommand{\tightlist}{%
  \setlength{\itemsep}{0pt}\setlength{\parskip}{0pt}}

% 
% set default figure placement to htbp
\makeatletter
\def\fps@figure{htbp}
\makeatother


\title{\textbf{Exercices cours de Secu sur les clouds}}
\author{Clément \textsc{Chavanon} \and Romain \textsc{Hu} \and Romain
\textsc{Hubert} \and Maxime \textsc{Grimaud} \and Volodia
\textsc{Parol-Guarino} \and 
 \\ Encadrant : Pascal \textsc{Garcia}}

% \author{Francesco \textsc{Bariatti} \and Adrien \textsc{Gasté} \and Mikael \textsc{Le} \and Romain \textsc{Lebouc}
%         \\
%         Encadrant : Pascal \textsc{Garcia}}

\date{2019-2020}

\begin{document}
\maketitle
\begin{abstract}
Créer de toutes pièces une IA sur le jeu des pingouins. Ce jeu est un
jeu de stratégie en plateau, sa principale caractéristique vient de ses
cases hexagonales. De plus ce jeu réagit très bien lorsque soumis à une
IA de type \emph{Monte-Carlo Tree Search}, que nous avons codé. Le
second défi de ce projet est également sa plateforme cible : exécuter le
code de l'interface et de l'IA dans un navigateur moderne. Pour cela
nous utilisons \texttt{Emscripten} qui nous permet de compiler notre IA
en \texttt{WebAssembly} et d'atteindre des performances proches du
natif. Quant au \emph{frontend}, c'est une application classique
\texttt{angular}.

\end{abstract}

{
\tableofcontents
}
\hypertarget{introduction}{%
\section*{Introduction}\label{introduction}}
\addcontentsline{toc}{section}{Introduction}

\hypertarget{le-jeu-des-pingouins}{%
\subsection{Le jeu des pingouins}\label{le-jeu-des-pingouins}}

Le jeu des pingouins est un jeu de stratégie sur plateau de 4 joueurs.
Le plateau contient 60 cases hexagonales sur lesquelles se trouvent 1 à
3 poissons.

En début de partie, chaque joueur place un certain nombre de pingouins
(de 2 à 4 suivant le nombre de joueurs) sur le plateau. A chaque tour,
le joueur doit, si possible, bouger l'un de ses pingouins. Les
mouvements de ceux se font sur en ligne droite depuis les 6 faces de la
case hexagonale sur laquelle il se trouve. Il ne peut passer par des
trous ou au-dessus d'autres pingouins, peu importe leur couleur. Une
fois le mouvement achevé, la case de départ est retiré du plateau. Le
joueur peut alors incrémenter du nombre de poisson qu'il y avait sur
cette case son score.

Le jeu se termine lorsque plus aucun pingouins ne peut se déplacer. Le
joueur avec le plus de points (poissons) remporte la partie.

% % \bibliography{references}
% 
\bibliography{references}

\end{document}
